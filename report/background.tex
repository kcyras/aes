\chapter{Background}

\section{Makeshift Schedules}

The simplistic definition of makeshift scheduling gives a good foundation for experimenting with argumentation. Makeshift schedules are defined by a $m\in\mathbb{N}$ independent machines and $n\in\mathbb{N}$ independent jobs. Let $\mathcal{M}=\{1,...,m\}$ be the set of machines and $\mathcal{J}=\{1,...,n\}$ be the set of jobs. Each job $i\in\mathcal{J}$, has an associated processing time $p_i\in\mathbb{R}_{\geq 0}$. All processing times are collectively denoted by a vector $\mathbf{p}$. A machine can only execute at most one job at any time. For a feasible schedule, each job is assigned to a machine non-preemptively. For some $i\in\mathcal{M}$, let $C_i$ be the completion time of the $i^\text{th}$ machine. Let $C_{max}$ be the total completion time. Let $\mathbf{x}\in\{0,1\}^{m\times n}$ be the assignment matrix that allocates jobs to machines. Formally, makeshift schedules are modelled as an optimisation problem:
\begin{align*}
	\min_{C_{max},\mathbf{C},\mathbf{x}}&C_{max}&\text{ subject to:}\\
	\forall i\in\mathcal{M}.\ &C_{max}\geq C_i\\
	\forall i\in\mathcal{M}.\ &C_i=\sum_{j\in\mathcal{J}}x_{i,j}\cdot p_j\\
	\forall j\in\mathcal{J}.\ &\sum_{j\in\mathcal{J}}x_{i,j}=1\\
	\forall i\in\mathcal{M},\ \forall j\in\mathcal{J}.\ &x_{i,j}\in\{0,1\}
\end{align*}

\begin{definition}
	A schedule $S$ is defined by its assignment matrix $\mathbf{x}$ such that $S=\mathbf{x}$.
\end{definition}

\begin{definition}
	A schedule $S$ is optimal iff $S$ achieves the minimal total completion time.
\end{definition}


\begin{definition}
	A machine $i\in\mathcal{M}$ is critical iff $C_i=C_{max}$.
\end{definition}

\begin{definition}
	A job $j\in\mathcal{J}$ is critical iff $j$ is allocated to a critical machine $i\in\mathcal{M}$ such that $x_{i,j}=1$.
\end{definition}

\begin{definition}
	A schedule satisfies the single exchange property (SEP) iff for any critical machine and any machine $i,i'\in\mathcal{M}$ and for all critical jobs $j\in\mathcal{J}$, $C_i-C_{i'}\leq p_j$
\end{definition}

\begin{definition}
	A schedule satisfies the pairwise exchange property (PEP) iff for any critical job and any job $j,j'\in\mathcal{J}$, if $p_j>p_{j'}$, then $C_i+p_{j'}\leq C_{i'}+p_j$.
\end{definition}

\begin{definition}
	A schedule $S$ is efficient iff $S$ satisfies SEP and PEP.
\end{definition}

\begin{theorem}
	Schedule efficiency is a necessary condition for optimality.
\end{theorem}

\section{User Fixed Decisions}

To accommodate practical applications of makeshift schedules, user positive and negative fixed decisions are introduced as an extension. In a hospital setting, positive fixed decisions capture patients exclusively allocated to a nurse while negative fixed decisions capture unavailable or incompatible nurses and patients. Let $D^-,D^+\subseteq\mathcal{M}\times\mathcal{J}$ be the negative and positive fixed decisions respectively. Let $D$ be the fixed decisions such that $D=(D^-,D^+)$.

\begin{definition}
	A schedule $S$ satisfies $D$ iff $\forall(i,j)\in D^-.\ x_{i,j}=0$ and $\forall (i,j)\in D^+.\ x_{i,j}=1$.
\end{definition}

\begin{definition}
	A fixed decision $D$ is unsatisfiable iff there does not exist a schedule $S$ such that $S$ satisfies $D$. 
\end{definition}

\begin{theorem}
	A fixed decision $D$ is unsatisfiable iff if the following necessary and sufficient conditions hold:
	\begin{itemize}
		\item$D^-$ and $D^-$ are disjoint.
		\item$\forall(i,j),(i',j')\in D^+.\ i=i'\lor j\neq j'$
		\item$\forall j\in\mathcal{J}.\ \exists i\in\mathcal{M}.\ (i,j)\not\in D^-$
	\end{itemize}
\end{theorem}

\section{Abstract Argumentation}

An abstract argumentation framework (AF) models the relation of attacks between arguments. Formally, an AF is a directed graph $(Args,\rightsquigarrow)$ where $Args$ is the set of arguments and $\rightsquigarrow$ is a binary relation over $Args$. For $a,b\in Args$, $a$ attacks $b$ iff $a\rightsquigarrow b$. Attacks are extended over sets of arguments, where $A\subseteq Args\rightsquigarrow b\in Args$ iff $\exists a\in A.\ a\rightsquigarrow b$. An extension $E$ is a subset of $Args$.

\begin{definition}
	An extension $E$ is conflict-free iff $\forall a,b\in E.\ a\not\rightsquigarrow\ b$.
\end{definition}

\begin{definition}
	An extension $E$ is stable iff $E$ is conflict-free and $\forall a\in Args\backslash E.\ E\rightsquigarrow a$
\end{definition}

\section{Existing Tools}

Due to the applicability of scheduling, there are many software specialise in staff rotas.

\section{Explanations}

\subsection{Notes}
\begin{enumerate}
	\item Conflicting negative and positive fixed decisions and explanation for their resolution
	\item Command line interface and improved graphical user interface with random, schedule display
	\item efficiency respects user fixed decisions
	\item explanations are sorted by reducible longest completion time. as m and n grows, the explanations grow. feasibility: $O(mn)$, efficiency: $O(mn^2)$, fixed decisions $O(mn)$
	\item Malle [112] explanations: find meanings, manage social interaction; learning
	\item why are explainations counterarguments: causes
	\item A tool explaining the concept of scheduling is superfluous while a tool explaining solutions concisely using mathematically-oriented may be discarded by inaccessible language. A challenge will be to strike a balance of possible explanations by targetting potential users.
\end{enumerate}

%https://ac.els-cdn.com/S0307904X12001837/1-s2.0-S0307904X12001837-main.pdf?_tid=8917b7ba-3293-40d2-a4d9-7d67b40ae499&acdnat=1546712183_67c96afaaedc7cd95cbff3dd83844cb6
