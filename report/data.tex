\chapter{Survey Responses}

\section{No Tool}
\label{notool}

\begin{tabular}{l l}
	Respondent & Time Taken \\
	\hline
	1 & 00:12:17 \\
	2 & 00:20:24 \\
	3 & 00:10:51 \\
	4 & 00:22:40 \\
	5 & 00:15:11 \\
	6 & 00:15:18 \\
	7 & 01:48:28 \\
	8 & 00:13:31 \\
	9 & 00:14:40 \\
	10 & 00:54:44 \\
\end{tabular}
\linespace
\begin{tabularx}{\linewidth}{c c X}
	Question & Respondent & Answer \\
	\hline
	1 & 1 & Switch C and D to machine 2 and F to machine 1. \\
	& 2 & Give job F to machine 1 and give jobs C and D to machine 2 \\
	& 3 & CD and F be swapped \\
	& 4 & Machine 1: A,B,E; Machine 2: C,D,F \\
	& 5 & Move F to machine 1 and C and D to machine 2 \\
	& 6 & Jobs F should be given to machine 1 and C and D should be given to machine 2 \\
	& 7 & move E up and CD down  \\
	& 8 & Move A, B to E. \\
	& 9 & 1. E,A,B; 2. F,C,D \\
	& 10 & Moving job F to machine 1, and jobs C and D to machine 2 will optimise the schedule. \\
	\hline
	2 & 1 & Switch G to machine 4 and B to machine 3. \\
	& 2 & Give job C to machine 2 \\
	& 3 & C be placed in machine 2 \\
	& 4 & Machine 1: A,E; Machine 2: D,C; Machine 3: G; Machine 4: B,F \\
	& 5 & Move C to machine 2 \\
\end{tabularx}
\begin{tabularx}{\linewidth}{c c X}
	& 6 & Jobs G and D should be swapped, schedule is not currently optimal \\
	& 7 & 1 - G; 2 - D, B; 3 - E, C; 4 -F, A \\
	& 8 & A,C and F work together. \\
	& 9 & move C to 2 \\
	& 10 & No this isn't optimal, having machine 2 handle jobs C and D, and machine 3 only handle job G would be an improvement. \\
	\hline
	3 & 1 & Move A and B to machine 3 and move E and F to machine 1. \\
	& 2 & Do all the shortest jobs first? \\
	& 3 & EF and AB be swapped \\
	& 4 & Machine 1: E,F; Machine 2: C,D; Machine 3: A,B \\
	& 5 & Assign it to machine 3 along with job B, move E and F to machine 1 \\
	& 6 & Swap E and F with A and B \\
	& 7 & 1 - B, C; 2- D, E; 3- F, A \\
	& 8 & Job A replace E \\
	& 9 & swap machine 1's jobs with machine 3 \\
	& 10 & Swap the jobs for machines 1 and 3. \\
	\hline
	4 & 1 & C and D, as this combination will add to a total time of 6, allowing A, B and E to also be added up to 6, thus creating the ideal possible schedule. \\
	& 2 & B and E constraint is better as it results in less time total for the machine that has to do those jobs compared to if a machine had to do both C and D \\
	& 3 & B and E as it means that the time split will be 6/6 \\
	& 4 & Machine 1: A,B,E; Machine 2: C,D \\
	& 5 & Both at the same time work well. A B and E on machine 1 and C and D on machine 2. That results in a an optimum 6 time overall. \\
	& 6 & If jobs C and D are assigned together then the total time will be shorter than if B and E are assigned together. So C and D result in a better schedule \\
	& 7 & 1 - C, D; 2 - A, B, E \\
	& 8 & Put Job B with machine 2 is better. \\
	& 9 & theyre the same \\
	& 10 & Either constraint should result in a better schedule. We pick the first constraint and move C to machine 2, and then move E to machine 1 (so now the second constraint is also satisfied). \\
	\hline
	5 & 1 & Switch D to machine 4 and F to machine 2. H can then be added to machine 2. \\
	& 2 & Move job E to machine 2 and give job H to machine 1 \\
	& 3 & 1 - AEF; 2 - (5 unit job); 3 - GC; 4 - BD \\
	& 4 & Machine 1: A,E; Machine 2: D,B; Machine 3: C,G; Machine 4: H,F \\
	& 5 & Move B to machine 2 and add H to machine 4 \\
	& 6 & A and D should be given to machine 1. H and F should be given to machine 2. C and G should be given to machine 3. B and E should be given to machine 4. The total time then remains unchanged at 6 \\
	& 7 & 1 - A, F; 2 - B. E; 3 - G, C; 4 - H, D \\
	& 8 & Move B machine 2 and new job in machine 4 \\
	& 9 & 1. H; 2. G, A; 3. E, C, F; 4. D, B \\
	& 10 & We will put job D in machine 3 and move job F to machine 2, and then also schedule job H to machine 2. This should keep the total time the same. \\
\end{tabularx}

\section{With Tool}