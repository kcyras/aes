\chapter{Survey Responses}

\section{Group 1: Without Explanations}
\label{group1data}

\begin{tabular}{l l}
	Respondent & Time Taken \\
	\hline
	1 & 00:12:17 \\
	2 & 00:20:24 \\
	3 & 00:10:51 \\
	4 & 00:22:40 \\
	5 & 00:15:11 \\
	6 & 00:15:18 \\
	7 & 01:48:28 \\
	8 & 00:13:31 \\
	9 & 00:14:40 \\
	10 & 00:54:44 \\
\end{tabular}
\linespace
\begin{tabularx}{\linewidth}{c c X}
	Question & Respondent & Answer \\
	\hline
	1 & 1 & Switch C and D to machine 2 and F to machine 1. \\
	& 2 & Give job F to machine 1 and give jobs C and D to machine 2 \\
	& 3 & CD and F be swapped \\
	& 4 & Machine 1: A,B,E; Machine 2: C,D,F \\
	& 5 & Move F to machine 1 and C and D to machine 2 \\
	& 6 & Jobs F should be given to machine 1 and C and D should be given to machine 2 \\
	& 7 & move E up and CD down  \\
	& 8 & Move A, B to E. \\
	& 9 & 1. E,A,B; 2. F,C,D \\
	& 10 & Moving job F to machine 1, and jobs C and D to machine 2 will optimise the schedule. \\
	\hline
	2 & 1 & Switch G to machine 4 and B to machine 3. \\
	& 2 & Give job C to machine 2 \\
	& 3 & C be placed in machine 2 \\
	& 4 & Machine 1: A,E; Machine 2: D,C; Machine 3: G; Machine 4: B,F \\
	& 5 & Move C to machine 2 \\
\end{tabularx}
\begin{tabularx}{\linewidth}{c c X}
	& 6 & Jobs G and D should be swapped, schedule is not currently optimal \\
	& 7 & 1 - G; 2 - D, B; 3 - E, C; 4 -F, A \\
	& 8 & A,C and F work together. \\
	& 9 & move C to 2 \\
	& 10 & No this isn't optimal, having machine 2 handle jobs C and D, and machine 3 only handle job G would be an improvement. \\
	\hline
	3 & 1 & Move A and B to machine 3 and move E and F to machine 1. \\
	& 2 & Do all the shortest jobs first? \\
	& 3 & EF and AB be swapped \\
	& 4 & Machine 1: E,F; Machine 2: C,D; Machine 3: A,B \\
	& 5 & Assign it to machine 3 along with job B, move E and F to machine 1 \\
	& 6 & Swap E and F with A and B \\
	& 7 & 1 - B, C; 2- D, E; 3- F, A \\
	& 8 & Job A replace E \\
	& 9 & swap machine 1's jobs with machine 3 \\
	& 10 & Swap the jobs for machines 1 and 3. \\
	\hline
	4 & 1 & C and D, as this combination will add to a total time of 6, allowing A, B and E to also be added up to 6, thus creating the ideal possible schedule. \\
	& 2 & B and E constraint is better as it results in less time total for the machine that has to do those jobs compared to if a machine had to do both C and D \\
	& 3 & B and E as it means that the time split will be 6/6 \\
	& 4 & Machine 1: A,B,E; Machine 2: C,D \\
	& 5 & Both at the same time work well. A B and E on machine 1 and C and D on machine 2. That results in a an optimum 6 time overall. \\
	& 6 & If jobs C and D are assigned together then the total time will be shorter than if B and E are assigned together. So C and D result in a better schedule \\
	& 7 & 1 - C, D; 2 - A, B, E \\
	& 8 & Put Job B with machine 2 is better. \\
	& 9 & theyre the same \\
	& 10 & Either constraint should result in a better schedule. We pick the first constraint and move C to machine 2, and then move E to machine 1 (so now the second constraint is also satisfied). \\
	\hline
	5 & 1 & Switch D to machine 4 and F to machine 2. H can then be added to machine 2. \\
	& 2 & Move job E to machine 2 and give job H to machine 1 \\
	& 3 & 1 - AEF; 2 - (5 unit job); 3 - GC; 4 - BD \\
	& 4 & Machine 1: A,E; Machine 2: D,B; Machine 3: C,G; Machine 4: H,F \\
	& 5 & Move B to machine 2 and add H to machine 4 \\
	& 6 & A and D should be given to machine 1. H and F should be given to machine 2. C and G should be given to machine 3. B and E should be given to machine 4. The total time then remains unchanged at 6 \\
	& 7 & 1 - A, F; 2 - B. E; 3 - G, C; 4 - H, D \\
	& 8 & Move B machine 2 and new job in machine 4 \\
	& 9 & 1. H; 2. G, A; 3. E, C, F; 4. D, B \\
	& 10 & We will put job D in machine 3 and move job F to machine 2, and then also schedule job H to machine 2. This should keep the total time the same. \\
\end{tabularx}

\section{Group 2: With Explanations but without Tool}
\label{group2data}

\begin{tabular}{l l}
	Respondent & Time Taken \\
	\hline
	1 & 00:09:41 \\
	2 & 00:01:13 \\
	3 & 00:03:16 \\
	4 & 00:06:25 \\
	5 & 00:08:06 \\
	6 & 00:11:41 \\
	7 & 00:03:25 \\
	8 & 00:09:22 \\
	9 & 00:05:26 \\
	10 & 00:09:14 \\
	11 & 00:12:02 \\
	12 & 00:11:02 \\
\end{tabular}
\\\\\\
\begin{tabularx}{\linewidth}{c c X}
	Question & Respondent & Answer \\
	\hline
	1 & 1 & Move job F to machine 1. move jobs C and D to machine 2. \\
	& 2 & Move E to 1 and A, B to 2 \\
	& 3 & Move C and D to 2. Move F to 1. \\
	& 4 & move F to 1; move C and D to 2 \\
	& 5 & Move C and D to 2 and move F to 1 \\
	& 6 & F to 1. C \& D to 2. \\
	& 7 & Move F to 1. Move C and D to 2. \\
	& 8 & 1: A, B, E 2: C, D, F \\
	& 9 & move F to 1; move C and D to 2 \\
	& 10 & Move F to start at machine 1, and move job C, D to machine 2 after E \\
	& 11 & A, B, and F to machine 1; C, D, and E to machine 2 \\
	& 12 & Move F to machine 1, move C and D to machine 2 \\
	\hline
	2 & 1 & Not optimal. Could be improved by moving job C to machine 2. \\
	& 2 & Move C to 2 \\
	& 3 &  \\
	& 4 & move C to machine 2 \\
	& 5 & No, move C to 2 \\
	& 6 & C to 2. Total time would be 5h. I do not think this can be reduced further. \\
	& 7 & No, because makespan can still be reduce by, e.g., moving Job C to machine 2. \\
	& 8 & Move C to machine 2 \\
	& 9 & move C to D \\
	& 10 & Jobs G and D can be swapped \\
	& 11 & No; Jobs D and G can be swapped with machines 2 and 3 to reduce by 1.0 \\
	& 12 & Swap G and D from machine 3 and 4 \\
	\hline
\end{tabularx}
\begin{tabularx}{\linewidth}{c c X}
	\hline
	3 & 1 & Move job A to machine 3. \\
	& 2 & ...and then do nothing. \\
	& 3 &  \\
	& 4 &  \\
	& 5 & Switch all the jobs of machine 1 and 3 \\
	& 6 & A \& B to 3. C \& D to 1. E \& F to 2. (permutation of the above schedule) \\
	& 7 &  \\
	& 8 & Swap the jobs of machines 1 and 3 \\
	& 9 & and move B to 3 next; move E and F to 1 \\
	& 10 & move job a to machine 3 then move job b to machine 3 as well, move E, F to machine 1 \\
	& 11 & Move job A to machine 3 and then job B to 3 and jobs E,F to machine 1 \\
	& 12 & Move A to machine 3 and move B to machine 3, move E and F to machine 1 \\
	\hline
	4 & 1 & Don't understand the question. \\
	& 2 &  \\
	& 3 &  \\
	& 4 & either \\
	& 5 & Both result in the same optimality \\
	& 6 & As hinted above, we can create a schedule lasting 6h (which I think is optimal) that respects both constraints. \\
	& 7 &  \\
	& 8 & Since C+D=A+B+E, the schedule can be optimal when both constraints are satisfied \\
	& 9 & A with B and E too; C and D together \\
	& 10 & C and D \\
	& 11 & Both of them simultaneously \\
	& 12 & Doesn't matter \\
	\hline
	5 & 1 & Assign job H to machine 4, move job B to machine 2. \\
	& 2 &  \\
	& 3 &  \\
	& 4 & H on 2; D on 4; F on 2 \\
	& 5 & Switch D and F \\
	& 6 & 1: A \& E; 2: B \& D; 3: C \& G; 4: F \& H; Total time = 6h \\
	& 7 &  \\
	& 8 &  \\
	& 9 & swap D and F \\
	& 10 & swap D and F and allocate H to machine 2 \\
	& 11 & Job F to machine 1; Job D to machine 4; and new job H to machine 2 \\
	& 12 & swap job D and F from machine 2 and 4 \\
\end{tabularx}

\newpage
\section{Group 3: With Tool}
\label{group3data}